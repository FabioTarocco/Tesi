\documentclass[final, 11pt]{article}

\usepackage[italian]{babel}
\usepackage{braket}
\usepackage{diagbox} 
%\usepackage{setspace}
\usepackage{tikz}

\usepackage{amsmath}
\usepackage{amsfonts}
\usepackage{algorithm}		%pseudocodice package
\usepackage{algorithmic}	%pseudocodice package

\usepackage[utf8]{inputenc}

\usepackage{listings}
\usepackage{xcolor}


\usepackage{graphicx}
\usepackage{pdfpages}
\title{Studio di framework per la generazione di fake images di vestiti: Analisi e confronto dei codici VITON e Virtual Try-On with Detail Carving}
\author{ Candidato: Fabio Tarocco VR421748}
\begin{document}
	\clearpage
	
\begin{titlepage}
	\centering
	\vspace*{\fill}
	{\scshape\LARGE Università degli Studi di Verona \par}
	\vspace{1.5cm}
	{\huge Studio di framework per la generazione di fake images di vestiti: Analisi e confronto dei codici VITON e Virtual Try-On with Detail Carving \par}
	\vspace{0.5cm}
	{\scshape  Corso di Laurea Triennale in Informatica \par}
	\vspace{1cm}
	{\Large\itshape Fabio Tarocco VR421748 \par}
	\vspace{1cm}
	\vspace{5cm}
	\vspace*{\fill}
	{\large 16 Ottobre 2020 \par}
\end{titlepage}
	\newpage
	\thispagestyle{plain} % empty
	\mbox{}
	\clearpage
	
	\tableofcontents
	\newpage
	\section{Introduzione al report}
	\subsection{Idea di base}
	asa
	\subsection{Cos'è un Virtual Try-On}
	Il segnale dato \textit{H(f)} si può vedere come la somma dei seguenti 3 segnali nel dominio frequenziale :
	\subsection {Obiettivi tirocinio}
	dasd
	
	\newpage
	\section{VITON}
	\subsection{Descrizione codice}
	dsad
	\subsection{Qualità output prodotti}	
	L'immagine viene acquisita come una matrice di pixel $ n \times m $, per semplicità assumiamo che essa sia in scala di grigi. Se l'immagine caricata non dovesse essere in \textit{grayScale} è possibile convertirla tramite \texttt{cvtColor( src, src\_gray, COLOR\_BGR2GRAY )}, funzione di \textit{convert color}.
	
	\newpage
	\section{VTO con Detail Carving}
	dsads
	\subsection{Descrizione codice}	
	dsa
	\subsection{Qualità output prodotti}
	Si valutano i segnali che vengono proposti nell'esercizio. Il primo è definito come un segnale con supporto illimitato e con frequenza massima 
	
	\newpage
	\section{Confronto risultati fra i due framework}	
	Si valutano i segnali che vengono proposti nell'esercizio. Il primo è definito come un segnale con supporto illimitato e con frequenza massima 
	
	\newpage
	\section{Variante ACGPN per testing}
	sa
	\subsection{ACGPN in breve}
	dsa
	\subsection{Risultati Demo}
	sa
	
\end{document}